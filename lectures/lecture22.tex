\chapter{Mar.~27 --- Introductory \texorpdfstring{$p$}{p}-adic Analysis}

\section{\texorpdfstring{$p$}{p}-adic Analysis}

\begin{lemma}
  If $K$ is a complete non-Archimedean
  field, then a series
  $\sum_{n = 1}^\infty a_n$ with $a_n \in K$
  converges in $K$ if and only if
  $a_n \to 0$.
\end{lemma}

\begin{proof}
  We want
  $|a_n + a_{n + 1} + \dots + a_m| < \epsilon$
  for $n, m \gg 0$, but
  \[
    |a_n + a_{n + 1} + \dots + a_m|
    \le \max\{|a_i|\}
  \]
  by the ultrametric inequality,
  so this happens if and only if
  $a_n \to 0$.
\end{proof}

\begin{remark}
  The whole idea of analytic continuation
  breaks in $p$-adic analysis because the
  space is not even connected. In fact,
  discs of the same radius
  are either disjoint or equal.
\end{remark}

\begin{definition}
  For a power series
  $f(x) = \sum_{n = 0}^\infty a_n x^n$,
  define its \emph{radius of convergence}
  to be
  \[
    R = \frac{1}{\limsup_{n \to \infty} |a_n|^{1 / n}}.
  \]
\end{definition}

\begin{lemma}
  Let $f(x) = \sum_{n = 1}^\infty a_n x^n$
  be a power series and $R$ its
  radius of convergence.
  Then $f$
  converges on
  $\{z \in K : |z| < R\}$
  and diverges on
  $\{z \in K : |z| > R\}$.
  On $\{z \in K : |z| = R\}$,
  $f$ either converges everywhere or
  nowhere.
\end{lemma}

\begin{lemma}
  Let $b_{i, j} \in K$.
  If for all $\epsilon > 0$, there exists
  $M$ such that $|b_{i, j}| < \epsilon$
  when $\max\{i, j\} \ge M$, then
  the series $\sum_i \sum_j b_{i, j}$
  and $\sum_j \sum_i b_{i, j}$
  both converge and
  \[
    \sum_i \sum_j b_{i, j}
    = \sum_j \sum_i b_{i, j}.
  \]
\end{lemma}

\begin{theorem}[Strassmann's theorem]
  Let $f(x) = \sum_{n = 0}^\infty a_n x^n \ne 0$
  and suppose that $|a_n| \to 0$
  (i.e. $f$ converges on
  $D = \{|x| \le 1\}$). Define
  $N = N(f)$ by $|a_N| = \max_n |a_n|$
  and $|a_n| < |a_N|$ for $n > N$.
  Then $f(x)$ has at most $N$ zeroes in $D$.
\end{theorem}

\begin{proof}
  The proof is by induction on $N$.
  First consider the case $N = 0$, and
  suppose otherwise that $f(\alpha) = 0$
  for some $\alpha \in D$. Then we have
  \[
    |a_0|
    = \left|-\sum_{n \ge 1} a_n \alpha^n\right|
    \le \max_{n \ge 1} |a_n| |\alpha|^n
    \le \max_{n \ge 1} |a_n
    < |a_0|,
  \]
  which is a contradiction.
  Now let $N \ge 1$. Suppose
  $f(\alpha) = 0$ for some $\alpha \in D$.
  Then for any $\beta \in D$,
  \[
    f(\beta)
    = f(\beta) - f(\alpha)
    = \sum_{n \ge 1} a_n (\beta^n - \alpha^n)
    = (\beta - \alpha)
    \underbrace{\sum_{n \ge 1} \sum_{j = 0}^{n - 1} a_n \beta^j \alpha^{n - 1 - j}}_{g(\beta)}.
  \]
  Then $g(x) = \sum_{j = 0}^\infty b_j x^j$,
  where $b_j = \sum_{t = 0}^\infty a_{j + 1 + t} \alpha^t$.
  Then $N(g) = N(f) - 1$, so by induction
  $g$ has at most $N - 1$ zeroes in
  $D$, which implies that $f$ has at most $N$ zeroes in $D$.
\end{proof}

\begin{remark}
  We will often be interested in the
  case $K = \Q_p$ and $D = \Z_p$.
  Note that $\Z \subseteq \Z_p$
  and $\Z_p$ is
  bounded in the $p$-adic topology.
\end{remark}

\begin{remark}
  Strassmann's theorem is a special
  case of \emph{Newton polygons}.
\end{remark}

\begin{definition}
  For an integer $n \ge 0$ and
  $t \in \Q_p$, define
  the \emph{binomial coefficients}
  by $\binom{t}{0} := 1$ and
  \[
    \binom{t}{n}
    = \frac{t(t - 1) \cdots (t - n + 1)}{n!}, \quad n > 0.
  \]
\end{definition}

\begin{lemma}
  If $t \in \Z_p$, then
  $\binom{t}{m} \in \Z_p$.
\end{lemma}

\begin{proof}
  We have $\binom{t}{n} \in \Z$ for
  all $t \in \Z$ and
  $\binom{t}{m}$ is continuous in $t$.
  Since $\Z$ is dense in $\Z_p$,
  $|\binom{t}{m}|_p \le 1$ for all
  $t \in \Z$ implies that
  $|\binom{t}{m}|_p \le 1$ for all
  $t \in \Z_p$.
\end{proof}

\begin{theorem}[$p$-adic binomial theorem]
  Consider $(1 + x)^t$.
  \begin{enumerate}
    \item If $t \in \N$, then
      $(1 + x)^t = \sum_{n = 0}^t \binom{t}{n} x^n$.
      If $t \in \Z$, then
      $(1 + x)^t = \sum_{n = 0}^\infty \binom{t}{n} x^n$
      for $x \in p \Z_p$
      (i.e. $|x|_p < 1$).
    \item If $|x|_p < 1$, then there
      exists $\Phi_x(t) = \sum_{n = 0}^\infty \gamma_n t^n \in \Q_p \llbracket t \rrbracket$
      converging for all $t \in \Z_p$
      such that $\Phi_x(t) = (1 + x)^t$
      for all $t \in \Z$.
  \end{enumerate}
\end{theorem}

\begin{proof}
  (2) Suppose first that $t \in \N$. Then
  we have
  \[
    (1 + x)^t
    = \sum_{n = 0}^\infty t(t - 1) \cdots (t - n + 1) \frac{x^n}{n!}.
  \]
  Since $|x|_p \le 1 / p$, we have
  $|x^n / n!|_p \to 0$ as $n \to \infty$, so
  we can rearrange the series to get
  \[
    (1 + x)^t = \sum_{n = 0}^\infty \gamma_n t^n
  \]
  where $\gamma_n = \gamma_n(x) \in \Q_p$
  is independent of $t$ and
  $|\gamma_n|_p \to 0$.
  See the notes for the case $n < 0$.
\end{proof}

\begin{exercise}
  Prove the following:
  \begin{enumerate}
    \item $\displaystyle \log_p(1 + x) = \sum_{n = 1}^\infty (-1)^{n + 1} \frac{x^n}{n}$
      converges if and only if $|x|_p < 1$.
    \item $\displaystyle \exp_p(x) = \sum_{n = 0}^\infty \frac{x^n}{n!}$
      converges if and only if
      $|x|_p < p^{-1 / (p - 1)}$
      (i.e. $v_p(x) \ge 1 / (p - 1)$).
  \end{enumerate}
\end{exercise}

\begin{remark}
  Note that $\exp : \C \to \C^\times$ is
  entire, while $\exp_p$ is not.
\end{remark}

\section{Applications to Diophantine Equations}

\begin{example}
  We show that the only integer
  solution to $x^3 - 11y^3 = 1$ is
  $(1, 0)$.

  First note that if
  $x^3 - 11 y^3 = 1$, then
  $x - y \sqrt[3]{11}$ is a unit of
  norm $1$ in $\Z[\sqrt[3]{11}]$. Let
  $\alpha = \sqrt[3]{11}$. The
  fundamental unit in
  $\Z[\alpha]$ is
  $u = 18\alpha^2 + 40\alpha + 89$, and
  $v = 1 / u = -2\alpha^2 + 4\alpha + 1$.
  Since every unit of $\Z[\alpha]$
  with norm $1$ is a power of $v$,
  we can write $x - y \sqrt[3]{11} = v^n$
  for some $n \in \Z$. We will show $n = 0$.

  Note that $19$ splits completely
  in $K = \Q(\alpha)$, so equivalently,
  $x^3 - 11$ has $3$ distinct roots
  modulo $19$. So by Hensel's lemma,
  we get $\alpha_1, \alpha_2, \alpha_3 \in \Z_{19}$
  satisfying $\alpha_i^3 = 11$.
  One can compute that
  \begin{align*}
    \alpha_1 &= -3 + 5 \cdot 19 + O(19^2), \\
    \alpha_2 &= -2 + 8 \cdot 19 + O(19^2), \\
    \alpha_3 &= 5 + 6 \cdot 19 + O(19^2).
  \end{align*}
  The $\alpha_i$ give $3$ embeddings
  $\psi_i : K \to \Q_{19}$.
  Let $v_i = \psi_i(v)$, which one can compute
  to be
  \begin{align*}
    v_1 &= 9 + 2 \cdot 19 + O(19^2), \\
    v_2 &= 4 + 0 \cdot 19 + O(19^2), \\
    v_3 &= 9 + 6 \cdot 19 + O(19^2).
  \end{align*}
  Since $x - y \sqrt[3]{11} = v^n$,
  we have $x - y \alpha_i = v_i^n$ for
  $i = 1, 2, 3$, so
  \[
    \alpha_1 v_1^n + \alpha_2 v_2^n + \alpha_3 v_3^n
    = \sum_i (x\alpha_i - y \alpha_i^2)
    = 0,
  \]
  where we get $\alpha_1 + \alpha_2 + \alpha_3 = \alpha_1^2 + \alpha_2^2 + \alpha_3^2 = 0$
  by looking at the coefficients of
  $x^3 - 11^j$. We continue the
  rest of the argument next class.
\end{example}
